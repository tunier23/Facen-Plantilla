
%------Sección 1------------%
\section{Creando trabajos en \LaTeX}

Hoy hablaremos de la energía cuántica y las frecuencias ideales en las cuales nos conviene vibrar.

%Subsección 1.1%
\subsection{Ejemplo sencillo}

Con esta plantilla podrás crear varios tipos de documentos y tendrás varios recursos para utilizar. 

\newp Con el comando \verb|\newp| podrás crear nuevos párrafos, con el comando  \verb|\newpage| nuevas páginas. También añadir pies de páginas\footnote{Pie de página fachera} con el comando \verb|\footnote{texto}|.

%Subsección 1.2%
\subsection{Insertando figuras}
Insertar figuras es muy sencillo, se utiliza el entorno \verb|figure|, un ejemplo de como añadirlo es con el código
\verbatiminput{ejemplos/imagen}

\newp \verb|[!htbp]| hará que la imagen aparezca en el lugar que lo añadimos, \verb|\centering| hará que la imagen esté centrada, \verb|\includegraphics[scale=valor]{ruta}| en scale se especifica el tamaño de la imagen y en ruta, la ubicación, por ejemplo \verb|img/grafico1.png| \verb|\caption{Descripción chida}| aquí va la descripción de la imagen en caso de que la tenga y \verb|label{fig:etiqueta}| aquí colocas una etiqueta en caso de que quieras referenciar la imagen (ej., Fig. \ref{fig:figura1}) .

\begin{figure}[!htbp]
    \centering
    \includegraphics[scale=0.28]{img/grafico1.jpg}
    \caption{Descripción chida}
    \label{fig:figura1}
\end{figure}

\newpage
%Sección 2%
\section{Tipos de trabajos}
Este trabajo soporta dos tipos: resumen y ejercitario.

%Subsección 2.1%
\subsection{Resumen}
En el modo resumen el trabajo adoptará cierta forma, como la que estás leyendo ahora, la diferencia es el enumerado de las imágenes, ecuaciones, etc. que se enumeran por las secciones.

%Subsección 2.2%
\subsection{Ejercitario}
En el modo ejercitario se asume que no habrá secciones por lo que las imagenes, ecuaciones, etc. se enumeran de acuerdo al número de problema.


\section{Entornos propios}
Hay algunos entornos propios de esta plantilla, por ejemplo, el entorno problema 

\subsection{El entorno problema-solución}
Ideal para entorno de problemas (xd), por ejemplo, el siguiente código genera 
\verbatiminput{ejemplos/problema}

\begin{problema}
    Sean $x, y, z, n$ un números enteros y $n \ge 3$, demuestre que se cumple la siguiente igualdad:
    $$ x^n + y^n = z^n $$
\end{problema}

Acompañado de su solución o demostración, como gusten.
\verbatiminput{ejemplos/demostracion}
\begin{proof}
    La demostración es trivial y se deja como ejercicio al lector.
\end{proof}

\verbatiminput{ejemplos/solucion}
\begin{solucion}
    Preguntenle a Andrew Wiles.
\end{solucion}

\newpage
\subsection{Entornos matemáticos}

\begin{definicion}
    Este es el entorno Definición que se utiliza con el comando \verb|definicion|.
\end{definicion}

\begin{proposicion}
    Este es el entorno Proposición que se utiliza con el comando \verb|proposicion|.
\end{proposicion}

\begin{lema}
    Este es el entorno Lema que se utiliza con el comando \verb|lema|.
\end{lema}

\begin{axioma}
    Este es el entorno Axioma que se utiliza con el comando \verb|Axioma|.
\end{axioma}

\begin{teorema}
    Este es el entorno Teorema que se utiliza con el comando \verb|teorema|.
\end{teorema}

\begin{corolario}
    Este es el entorno Corolario que se utiliza con el comando \verb|corolario|.
\end{corolario}


\section{Ecuaciones}
En esta sección veremos algunos entornos para ecuaciones, delimitadores y notaciones.

\subsection{Entorno ecuación}
Un entorno simple para ecuaciones es el siguiente \ref{eqn:ecuacion1}
\verbatiminput{ejemplos/ecuacion1}
\begin{equation}
    x^2 + y^2 = z^2 
    \label{eqn:ecuacion1}
\end{equation}

Otra opción más \textit{elegante} es utilizar el siguiente comando:
\verbatiminput{ejemplos/ecuacion2}
\begin{equation}
    \tcboxmath{ 
    x^2 + y^2 = z^2
    }
    \label{eqn:ecuacion2}
\end{equation}

\subsection{Delimitadores}
Los delimitadores propios de la plantilla son
\begin{itemize}
    \item \verb|\parentesis{x^2}|  $\parentesis{x^2}$
    \item \verb|\corchete{x^2}|  $\corchete{x^2}$
    \item \verb|\llave{x^2}|  $\llave{x^2}$
    \item \verb|\modulo{x^2}|  $\modulo{x^2}$
    \item \verb|\absoluto{x^2}|  $\absoluto{x^2}$
\end{itemize}

\newpage
\subsection{Notaciones}
Así también se incorporan notaciones matemáticas, algunas de ellas son
\begin{itemize}
    \item \verb|\vec{v}| notación para vectores $\vec{v}$.
    \item \verb|\uvec{e}| notación para vectores unitarios $\uvec{e}$.
    \item \verb|\expval{x,y,z}| notación para vectores en componentes $\expval{x,y,z}$.
    \item \verb|\Eval{F(x)}{a}{b}| evaluar expresiones $\int_a^b f(x)\, dx=\Eval{F(x)}{a}{b}$.
\end{itemize}


\section{Integración con otros paquetes}

\subsection{Tikz}
Esta plantilla contiene varias librerías del paquete tikz, para más información \href{https://www.bu.edu/math/files/2013/08/tikzpgfmanual.pdf}{aquí}.

\newp Con tikz podrás hacer muchos gráficos, diagramas, etc. algunos ejemplos:

 \begin{figure}[H]
     \centering
     \begin{tikzpicture}[scale=0.65]
        \def\clr{olive}
        
        \def\a{3} \def\b{9} 
        
        \newcommand{\xt}[1]{(\b-\a)*cos(#1)+\a*cos((\b-\a)*#1/\a}
        \newcommand{\yt}[1]{(\b-\a)*sin(#1)-\a*sin((\b-\a)*#1/\a}

        
        \draw[-latex] (-\b-1,0) -- (\b+1,0) node[right] {$x_1$};
        \draw[-latex] (0,-\b-1) -- (0,\b+1) node[above] {$x_2$};
        

        \draw[\clr,thick] (0,0) circle (\b);
        

        \draw[\clr,dashed,very thin] (0,0) circle (\b-\a);

        \draw[line width=2pt,orange!80!red] plot[samples=100,domain=0:\a*360,smooth,variable=\t] ({\xt{\t}},{\yt{\t}});

        
        \def\t0{40}
        \draw[\clr!50!black] (\t0:\b-\a) circle (\a);
        \draw[purple,fill] ({\xt{\t0}},{\yt{\t0}}) circle (2pt) -- (\t0:\b-\a) circle (1pt) node [midway, sloped, above right] {\small $r $} -- (0,0) circle (1pt) node [midway, sloped, above left] {\scriptsize $r_0 - r$} node [below right] {$O$};
        
        \draw (0,0) -- (\b+1,\b-0.55);
         \draw[purple,fill] (\b-2.15,\b-3.17) circle (2pt) node[above] {$B$};
        \draw ({\xt{\t0}},{\yt{\t0}}) node[below] {$P$};
        \draw[purple,fill] (\b,0) circle (2pt) node[above right] {$D$};
        
         \coordinate (D) at (\b,0);
          \coordinate (O) at (0,0);
          \coordinate (B) at (\b-2.15,\b-3.17);
          \coordinate (P) at ({\xt{\t0}},{\yt{\t0}});
          
        
        \pic [draw, -latex, "$\theta$", angle eccentricity=1.5, angle radius=0.7cm] {angle = D--O--B};
        
         \draw[purple,fill] (4.6,3.9) circle (2pt);
          \coordinate (A) at (4.6,3.9);
          
          
          \draw[purple,fill] (7.6,3.9) circle (2pt);
          \coordinate (C) at (7.6,3.9);
          
          \pic [draw, -latex, "$\theta$", angle eccentricity=1.5, angle radius=0.6cm] {angle = C--A--B};
          
          \draw (A) -- (C);
          
          \draw (5.4,-6.8) node[above] {$C_0$};
          
          \pic [draw, purple, latex-, "$\phi$", angle eccentricity=1.2, angle radius=1.1cm] {angle = P--A--B};
          
          \draw (4.5,0) node[below] {$r_0$};
        
    \end{tikzpicture}
     \caption{Hipocicloide.}
     \label{fig:5.1}
 \end{figure} 
 \begin{figure}[H]
    \centering
    \begin{tikzpicture}[x=1cm, y=1cm, z=-0.6cm]
    % Axes
    \draw [->] (0,0,0) -- (5,0,0) node [at end, right] {$y$};
    \draw [->] (0,0,0) -- (0,5,0) node [at end, left] {$z$};
    \draw [->] (0,0,0) -- (0,0,5) node [at end, left] {$x$};

    % Vectors
    
    \draw [-latex, thick] (0,0,0) -- (4,4,4);

    %\draw [loosely dashed] (0,0,3) -- (3.7,0,2.7);

   \draw [-latex] (0,0,0) -- (4,0,4);
   
    \node [left] at (2.3,0,2.4) {$\rho$};
    
    \draw [loosely dashed] (4,0,4) -- (4,4,4);
    \draw [loosely dashed] (4,0,0) -- (4,0,4);
    \draw [loosely dashed] (0,0,4) -- (4,0,4);
    % Labels
    \node [right] at (5.2,4.1,4.6) {$P(\rho,\theta,z)$};
    
    
    \node [left] at (3,3.2,3.2) {$\vec{r}$};
    
    \filldraw (4,4,4) circle (1pt);
    
    \coordinate (A) at (0,0,1);
    \coordinate (B) at (0,0,0);
    \coordinate (C) at (1,0,1);
    \pic [draw, -, "$\theta$", angle radius=0.8cm, angle eccentricity=1.4] {angle = A--B--C};
    
    \draw [-latex, very thick]  (4,4,4) --  (5,4,5);
    
    \draw [-latex, very thick]  (4,4,4) --  (4.5,4,3.5);
    
    \draw [-latex, very thick]  (4,4,4) --  (4,5,4);
    
    \node [below] at (5,4,5) {$\uvec{r}$};
    
    \node [below] at (4.5,4,3.5) {$\uvec{\theta}$};
    
    \node [above] at (4,5,4) {$\uvec{z}$};
    
\end{tikzpicture}
    \caption{Punto $P(\rho, \theta, z)$ en coordenadas cilíndricas.}
    \label{fig:5.2}
\end{figure}
\begin{figure}[H]
\centering
\begin{tikzpicture}[%
angle eccentricity=1.2,
ball/.style={circle, inner sep=0pt, minimum size=2mm, draw=black, fill=terracota!50,  label=right:$m$}]

\draw[thick, black] (-2,0) --(2,0);
\draw[black] (0,0) coordinate (b0) foreach \i [count=\ni] in {-70,-55} {--++(\i:2cm) node[midway,auto]{$L$} node[ball] (b\ni) {}};

\foreach \i [count=\auxi] in {b0,b1}{
    \draw[dashed,black] (\i)--++(-90:1.8cm) coordinate[pos=.75] (aux\auxi);
} 

\draw pic["$\theta_1$", draw, black, angle radius=1.2cm] {angle=aux1--b0--b1};
\draw pic["$\theta_2$", draw, black, angle radius=1.2cm] {angle=aux2--b1--b2};

\end{tikzpicture}
\caption{El temible péndulo doble.}
\label{fig:5.3}
\end{figure}
\begin{figure}[H]
\centering
\begin{tikzpicture}
\coordinate (origen) at (0,0);
\coordinate (i) at (3,0);
\coordinate (imax) at (3,3);

\draw[->] (-1,0)--(4,0) node[right]{$\Re{I}$};
\draw[->] (0,-1)--(0,4) node[above]{$\Im{I}$};
\draw (0,0) node[below left]{$0$};
\draw[line width=2pt,blue,-stealth](0,0)--(3,3) node[anchor=south west, above left]{${I_0}$};
\draw[line width=2pt,lila,-stealth](0,0)--(3,0) node[anchor=south west, below left]{${I = I_0 \cos{\omega t}}$};
\draw[dashed] (3,0) -- (3,3);

\pic [draw, ->, "$\omega t$", angle radius=1cm, angle eccentricity=1.4] {angle = i--origen--imax};

\draw [celeste, dashed, <-, thick] plot [smooth, tension=0.5] coordinates {   (1.5,1.8) (1.3,3) (2,4) (3,4.5)} [anchor= west] node{\emph{\textcolor{celeste}{La longitud del
fasor es igual a la corriente máxima $I_0$.}}};

\draw [black, ->, thick] plot [smooth, tension=1] coordinates {  (3.6, 2.8) (3.3,3.3) (2.8, 3.6)};

\draw[thick] (3.6,3.25) node [above, sloped] (TextNode) {$\omega$};

\draw [celeste, dashed, <-, thick] plot [smooth, tension=0.5] coordinates {   (3.7,3.1) (4.5,3.1) } [anchor= west] node{\emph{\textcolor{celeste}{El fasor gira con
frecuencia $f$ y rapidez angular $\omega = 2 \pi f$.}}};

\draw [celeste, dashed, <-, thick] plot [smooth, tension=1] coordinates {   (2.5,0.31) (3,0.77) (3.5,1.05) (4,1.27) (5,1.61) } [anchor= west] node{\emph{  \begin{tabular}{l}
     \textcolor{celeste}{La proyección del fasor sobre el eje horizontal  }  \\
     \textcolor{celeste}{en el  tiempo $t$ es igual a la corriente $I$en ese}  \\
     \textcolor{celeste}{ instante: $I=I_0 \cos{\omega t}$.}
\end{tabular}  }};

\end{tikzpicture}
\caption{Diagrama fasorial de la corriente $I$.}
\label{fig:fasorsencillo}
\end{figure}



\subsection{PrimeTree}
Con este comando se crea un árbol de factores primos, por ejemplo \verb|\PrimeTree{23100}| genera lo siguiente:       

\PrimeTree{23100}

\subsection{CircuiTikz}
Ideal para crear circuitos, más información sobre el paquete \href{https://texdoc.org/serve/circuitikz/0}{aquí}.

%-----Circuito con tikz-----------%
\begin{figure}[H]
     \centering
      \begin{circuitikz}[american, cute inductors]
     \draw (0,0) to[sV=$\varepsilon (t)$,fill=yellow] (0,4);
     \draw (0,3.01) to[short, f>_=$I(t)$] (0,3.02);
     \draw (0,4) to[R, l=$R$, *-*] (4,4) to[L, l=$L$, *-*] (8,4) ;
     \draw (2,3) node[above]{$\varepsilon_R (t)$};
     \draw (6,3) node[above]{$\varepsilon_L (t)$};
    \draw (8,4) to[C, l=$C$, *-*] (8,0);
    \draw (9,2) node[right]{$\varepsilon_C (t)$};
    \draw (8,0) -- (0,0);
    \draw (8,0) to[short, -o] (10,0);
    \draw (8,4) to[short, -o] (10,4);
    \end{circuitikz}
     \caption{Circuito RLC en serie.}
     \label{fig:rlcEQ}
 \end{figure}













